\documentclass[conference]{IEEEtran}
\IEEEoverridecommandlockouts
% The preceding line is only needed to identify funding in the first footnote. If that is unneeded, please comment it out.
\usepackage{cite}
\usepackage{amsmath,amssymb,amsfonts}
\usepackage{algorithmic}
\usepackage{graphicx}
\usepackage{textcomp}
\usepackage{xcolor}
\usepackage[brazil]{babel}
\usepackage[utf8]{inputenc}
\newcommand\tab[1][1cm]{\hspace*{#1}}
\usepackage{hyperref}
\def\BibTeX{{\rm B\kern-.05em{\sc i\kern-.025em b}\kern-.08em
    T\kern-.1667em\lower.7ex\hbox{E}\kern-.125emX}}
\begin{document}

\title{Implementação de detecção de faces em Python utilizando a biblioteca DLIB}


\author{\IEEEauthorblockN{José B. M. Trineto}
\and
\IEEEauthorblockN{Werberson P. Silva }
\\
Universidade de Brasília – Departamento de Engenharia Elétrica
}

\maketitle

\begin{abstract}
 Redes neurais artificiais (ANNs do inglês \textit{Artificial Neural Networks}) tem sido a base para diversos modelos aplicados na área de visão computacional (CV do inglês \textit{Computer Vision}). Dentre as ANNs, podemos destacar a rede neural convolucional (CNN do inglês \textit{Convolutional Neural Network}) utilizada para a construção de algoritmos de detecção de faces. Neste artigo, será demonstrado o desenvolvimento de uma aplicação para o reconhecimento de faces humanas em imagens, utilizando a linguagem de programação Python em conjunto com a biblioteca DLIB. 
\end{abstract}

\end{document}
