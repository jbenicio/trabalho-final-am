\documentclass[conference]{IEEEtran}
\IEEEoverridecommandlockouts
% The preceding line is only needed to identify funding in the first footnote. If that is unneeded, please comment it out.
\usepackage{cite}
\usepackage{amsmath,amssymb,amsfonts}
\usepackage{algorithmic}
\usepackage{graphicx}
\usepackage{textcomp}
\usepackage{xcolor}
\usepackage[brazil]{babel}
\usepackage[utf8]{inputenc}
\newcommand\tab[1][1cm]{\hspace*{#1}}
\usepackage{hyperref}
\def\BibTeX{{\rm B\kern-.05em{\sc i\kern-.025em b}\kern-.08em
    T\kern-.1667em\lower.7ex\hbox{E}\kern-.125emX}}

\title{Estudo e implementação de detecção de faces em imagens utilizando Python e a biblioteca DLIB}
\author{
    \IEEEauthorblockN{José B. M. Trineto \\ Werberson P. da Silva}
    \IEEEauthorblockA{
        \textit{Universidade de Brasília - Departamento de Engenharia Elétrica}
    }
}

\begin{document}
    \maketitle
    \begin{abstract}
        Redes neurais artificiais (ANNs do inglês \textit{Artificial Neural Networks}) tem sido a base para diversos modelos aplicados na área de visão computacional (CV do inglês \textit{Computer Vision}). Dentre as ANNs, podemos destacar as redes neurais convolucionais (CNNs do inglês \textit{Convolutional Neural Networks}) utilizadas para a construção de algoritmos de detecção de padrões em imagens. Neste artigo, será apresentado o estudo de CNNs aplicadas a CV e demonstrado o desenvolvimento de uma aplicação para o reconhecimento de faces humanas em imagens, utilizando a linguagem de programação Python em conjunto com a biblioteca DLIB. 
    \end{abstract}
     \begin{IEEEkeywords}
     	  Detecção Facial, Redes Neurais Convolucionais, Python, Biblioteca DLIB.
	\end{IEEEkeywords}
    \section{INTRODUÇÃO}
		A detecção facial é uma técnica de CV utilizada para a detecção de faces humanas em imagens.	Apesar de ser um procedimento simples para os homens, é uma tarefa com um alto grau de complexidade para computadores, visto que rostos podem variar em iluminação, cor, posicionamento e escala. Existem uma gama de aplicações para a detecção facial, como exemplos, o reconhecimento de faces humanas, o autofoco utilizado em câmeras digitais, dentre outras.
		
	    As redes neurais convolucionais são utilizadas para a maioria das soluções de CV, aplicadas a uma ampla variedade de tarefas, dentre elas, a detecção facial. As CNNs tem apresentado resultados satisfatórios para problemas relacionados ao reconhecimento de padrões em imagens, com um custo computacional menor se comparado a uma rede neural clássica, pois buscam explorar subestruturas presentes nas imagens para otimizar o processamento.
	    
		Atualmente, diversas linguagens de programação e ferramentas possuem implementações que utilizam redes neurais convolucionais para a resolução de problemas relacionados a detecção de faces. Dentre estas soluções, podemos destacar a biblioteca DLIB utilizada em Python. Esta biblioteca foi escrita em C++ e possui diversas vantagens em relação a outras implementações, tais como, o fácil aprendizado, a excelente documentação, a alta performance e etc. Este artigo tem como objetivo a implementação de uma aplicação de em Python utilizando a biblioteca DLIB para a detecção de faces humanas em imagens.
		
		Este artigo está organizado da seguinte forma: a Seção II irá apresentar um estudo sobre os principais conceitos relacionados as redes neurais convolucionais no processamento de imagem; na Seção III será apresentado a aplicação desenvolvida em Python para o reconhecimento de faces humanas em imagens; a Seção IV mostrará os resultados obtidos da aplicação desenvolvida e a Seção V irá apresentar as conclusões.   
		
	 \section{REDES NEURAIS CONVOLUCIONAIS}
	
		
\end{document}
