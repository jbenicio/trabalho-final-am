\documentclass[conference]{IEEEtran}
\IEEEoverridecommandlockouts
% The preceding line is only needed to identify funding in the first footnote. If that is unneeded, please comment it out.
\usepackage{cite}
\usepackage{amsmath,amssymb,amsfonts}
\usepackage{algorithmic}
\usepackage{graphicx}
\usepackage{textcomp}
\usepackage{xcolor}
\usepackage[brazil]{babel}
\usepackage[utf8]{inputenc}
\newcommand\tab[1][1cm]{\hspace*{#1}}
\usepackage{hyperref}
\def\BibTeX{{\rm B\kern-.05em{\sc i\kern-.025em b}\kern-.08em
    T\kern-.1667em\lower.7ex\hbox{E}\kern-.125emX}}

\title{Estudo e implementação de detecção de faces em imagens utilizando Python e a biblioteca DLIB}
\author{
    \IEEEauthorblockN{José B. M. Trineto \\ Werberson P. da Silva}
    \IEEEauthorblockA{
        \textit{Universidade de Brasília - Departamento de Engenharia Elétrica}
    }
}

\begin{document}
    \maketitle
    \begin{abstract}
        Redes neurais artificiais (ANNs do inglês \textit{Artificial Neural Networks}) tem sido a base para diversos modelos aplicados na área de visão computacional (CV do inglês \textit{Computer Vision}). Dentre as ANNs, podemos destacar as redes neurais convolucionais (CNNs do inglês \textit{Convolutional Neural Networks}) utilizadas para a construção de algoritmos de detecção de padrões em imagens. Neste artigo, será apresentado o estudo de CNNs aplicadas a CV e demonstrado o desenvolvimento de uma aplicação para o reconhecimento de faces humanas em imagens, utilizando a linguagem de programação Python em conjunto com a biblioteca DLIB. 
    \end{abstract}
    \section{INTRODUÇÃO}
\end{document}
